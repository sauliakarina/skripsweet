%!TEX root = ./template-skripsi.tex
%-------------------------------------------------------------------------------
% 								BAB I
% 							LATAR BELAKANG
%-------------------------------------------------------------------------------

\chapter{LATAR BELAKANG}

\section{Latar Belakang}
Universitas Negeri Jakarta (UNJ) sebagai salah satu institusi perguruan tinggi mempunyai peran dan fungsi mempersiapkan sumber daya manusia yang handal dan kompetitif sesuai bidangnya yang menjadi aset masyarakat, pemerintah dan bangsa , sehingga dapat memberikan kontribusi dalam pembangunan bangsa dan negara Indonesia \cite{BPAFMIPA}. Dengan demikian, UNJ diharapkan dapat menghasilkan lulusan berkompeten di bidangnya yang siap terjun ke jenjang pekerjaan dan masyarakat. Untuk menciptakan lulusan yang berkualitas tentunya suatu perguruan tinggi harus memiliki sistem pendidikan yang baik. Sistem pendidikan suatu program studi dikatakan baik jika lulusannya dibutuhkan oleh dunia kerja pengguna lulusan, sehingga untuk mencapai dalam taraf tersebut maka program studi harus tahu keinginan para pengguna lulusan \cite{EkoNursubiyantoro}. Untuk mengetahui tingkat relevansi (kesesuaian) antara kemampuan lulusan yang diperoleh melalui proses pendidikan di perguruan tinggi dengan kebutuhan dunia kerja dapat dilakukan upaya penelusuran alumni (\textit{tracer study}).

\textit{Tracer study} adalah studi penelusuran jejak alumni dilakukan setelah kelulusan dan bertujuan untuk mengetahui outcome pendidikan dalam bentuk transisi dari dunia pendidikan tinggi ke dunia kerja \cite{ExploringTS}. \textit{Tracer study} sangat berguna untuk evaluasi terhadap hasil pendidikan tinggi, relevansi dan sumber informasi bagi pemangku kepentingan dalam penentuan kebijakan salah satunya pengembangan kurikulum, serta kelengkapan persyaratan bagi akreditasi Dikti \cite{RistekdiktiPanduan}. Oleh karena itu, \textit{tracer study} menjadi sangat penting untuk menjadi bahan pertimbangan dalam penentuan kebijakan akademik, khususnya dalam penyusunan kurikulum dan penilaian akreditasi.

Sejak tahun 2011 Dikti telah mengembangkan sistem online untuk merintis kompilasi data \textit{tracer study} nasional mengenai transisi dan posisi pekerjaan lulusan perguruan tinggi di Indonesia. Hasil \textit{tracer study} yang kemudian dilaporkan ke Dikti akan membantu program Pemerintah dalam rangka memetakan kebutuhan dunia kerja dengan pembangunan pendidikan di Indonesia \cite{RistekdiktiWeb}. Namun baru sejumlah 675 data alumni UNJ yang tersedia pada sistem tersebut.

Berdasarkan website resmi Universitas Negeri Jakarta, UNJ memiliki delapan fakultas yang mengelola berbagai program studi baik pendidikan maupun non-pendidikan. Salah satu program studi yang ada di UNJ ialah Ilmu Komputer yang merupakan program studi non-pendidikan. Berdasarkan informasi yang didapat dari Koorprodi Ilmu Komputer, prodi Ilmu Komputer telah melakukan kegiatan \textit{tracer study} sejak prodi ini memiliki lulusan pertama pada semester 2016/2017. Pelaksanaan \textit{tracer study} tersebut dilakukan melalui pengiriman kuesioner melalui pos, \textit{e-mail} dan media sosial seperti \textit{whatsapp}. Bentuk pelaksanaan tersebut dirasa kurang efektif karena dapat membutuhkan waktu lama terhadap respon alumni. Saat itu, di UNJ memang belum ada sistem informasi \textit{tracer study} baik ditingkat Universitas, Fakultas, dan Prodi, semua kegiatan \textit{tracer study} dilakukan secara manual dengan mengirimkan \textit{e-mail} atau \textit{google docs}. Menurut Bapak Prasetyo selaku staf wakil rektorat bidang kemahasiswaan cara itu sangat manual dan tidak tervalidasi, artinya bisa saja ada kemungkinan orang lain yang bukan alumni bisa mengisi data-data \textit{tracer study} tersebut \cite{Rifqi}. 

Berdasarkan uraian di atas diadakan penelitian oleh Rifqi Syahirul Alim pada tahun 2019 dalam skripsinya berjudul “Perancangan dan Implementasi Sistem Informasi Penelusuran Alumni (\textit{Tracer Study}) Universitas Negeri Jakarta”. Penelitian tersebut telah menghasilkan sistem informasi penelusuran alumni yang dapat memudahkan pihak universitas maupun prodi dalam mengelola dan mengarsipkan data alumni.

Hasil analisis dari perbandingan sistem \textit{tracer study} yang telah ada dituangkan pada tabel perbandingan fitur [Lampiran A] dan diuraikan sebagai berikut. Sistem \textit{Tracer Study} yang dikembangkan oleh Dikti mempermudah setiap perguruan tinggi untuk melihat hasil atau laporan tracer study karena disediakan halaman yang menampilkan hasil \textit{tracer study} dalam bentuk grafik dan tabel yang informatif. Sistem ini menyajikan data \textit{tracer study} yang dapat dilihat oleh pengunjung tanpa harus login ke sistem. Kuesioner pada sistem ini telah disediakan dan distandarisasi oleh Dikti. Namun, data yang ada pada sistem tidak dapat diekspor, sehingga pihak perguruan tinggi akan kesulitan ketika akan membuat laporan. Selain itu, sistem tidak menyediakan kuesioner bagi pengguna lulusan. Informasi yang didapatkan dari Koorprodi Ilmu Komputer bahwa penelitian terhadap pengguna diperlukan untuk mengetahui bagaimana penilaian pengguna terhadap kompetensi lulusan dan kurikulum yang berjalan apakah sudah mencukupi dan relevan dengan kebutuhan dunia kerja saat ini. Selain itu, akreditasi perguruan tinggi juga membutuhkan informasi mengenai evaluasi kinerja lulusan dan umpan balik dari pengguna lulusan \cite{BorangIlkom}. Penelitian terkait telah dilakukan oleh Achmad Ghozaly et al dalam jurnalnya berjudul “Rancang Bangun Aplikasi \textit{Tracer Study} Berbasis Web pada Stikes Yayasan Rs. Dr. Soetomo Surabaya”. Pada penelitian tersebut selain kuesioner untuk alumni pada sistem juga disediakan kuesioner dan umpan balik yang diperuntukkan bagi pengguna alumni \cite{Ghozaly}. 

Sistem informasi \textit{tracer study} Universitas Negeri Jakarta menyediakan layanan tracer study di tingkat perguruan tinggi dan juga program studi. Admin baik universitas maupun prodi dapat mengelola form kuesioner sesuai dengan kebutuhan. Sistem tersebut dapat menampilkan hasil tracer study yaitu berupa grafik dan tabel yang juga informatif. Admin universitas dapat mengunggah data alumni dari seluruh prodi di UNJ. Alumni dapat melakukan pengisian form kuesioner secara online. Selain itu, hasil\textit{ tracer study} dapat diekspor ke dalam format \textit{excel} sehingga memudahkan pihak universitas maupun prodi dalam membuat laporan \textit{tracer study}. Namun, pada sistem tersebut belum tersedia kuesioner yang diperuntukkan bagi pengguna lulusan. Kemudian, baik prodi maupun universitas tidak dapat mengelola halaman beranda. Dimana pada beranda alumni terdapat kata pengantar yang tidak dapat disunting oleh admin melalui suatu tatap muka. Admin prodi tidak dapat melihat daftar alumni prodinya masing-masing. Selain itu, belum tersedianya halaman bagi pengunjung untuk dapat melihat data \textit{tracer study}. 

Berdasarkan uraian di atas akan dikembangkan Sistem \textit{Tracer Study} Program Studi Ilmu Komputer Fakultas Matematika dan Ilmu Pengetahuan Alam Universitas Negeri Jakarta berdasarkan analisis dan evaluasi dari perbandingan sistem \textit{tracer study} yang telah tersedia, yaitu Sistem \textit{Tracer Study} Dikti dan Sistem Informasi \textit{Tracer Study} UNJ. Pada penelitian ini diusulkan beberapa fitur, yaitu pertama berdasarkan penjelasan di atas peran pengguna alumni diperlukan untuk penilaian terhadap kompetensi lulusan, masukan bagi program studi, dan memenuhi kebutuhan akreditasi, maka pada pengembangan sistem ini akan disediakan akses bagi pengguna alumni untuk mengisi kuesioner. Pengunjung sistem dapat melihat daftar pengguna alumni sehingga dapat memfasilitasi pengunjung dalam mencari informasi pekerjaan di bidang Ilmu Komputer. Kedua, informasi yang didapat dari koorprodi Ilmu Komputer bahwa koorprodi membutuhkan akses terhadap sistem untuk mengakses data tracer yang diperlukan bagi akreditasi maupun penjaminan mutu internal. Ketiga, pada sistem akan dibuat suatu tatap muka untuk admin agar dapat mengelola kata pengantar tracer study pada beranda alumni sesuai kebutuhan. Terakhir, disediakan halaman data tracer yang dapat diakses oleh pengunjung sehingga dapat memfasilitasi masyarakat untuk lebih mengenal lulusan prodi Ilmu Komputer UNJ. Pengembangan sistem \textit{tracer study} ini tertuang pada penelitian yang berjudul \textbf{“Pengembangan Sistem Tracer Study Program Studi Ilmu Komputer FMIPA Universitas Negeri Jakarta”}.

\section{Identifikasi Masalah}
Dari uraian yang dikemukakan pada latar belakang, maka dapat diidentifikasi beberapa masalah sebagai berikut:
\begin{enumerate}
	\item Pada sistem \textit{tracer study} yang telah ada belum disediakannya kuesioner yang diperuntukkan bagi pengguna alumni.
	
	\item Pada Sistem \textit{Tracer Study} Dikti hasil tracer tidak dapat diekspor.
	
	\item Pada Sistem \textit{Tracer Study} UNJ tidak disediakan tatap muka (\textit{interface}) untuk mengelola kata pengantar tracer study, tidak adanya halaman pengunjung untuk melihat data \textit{tracer}, dan admin prodi tidak dapat melihat daftar lulusannya. 
	
\end{enumerate}

\section{Rumusan Masalah}
Adapun rumusan masalah berdasarkan pada latar belakang yang telah dipaparkan adalah sebagai berikut :
\begin{enumerate}
	\item Bagaimana konsep rancangan dari pengembangan Sistem \textit{Tracer Study} Program Studi Ilmu Komputer FMIPA UNJ ?
	
	\item Bagaimana implementasi rancangan pengembangan Sistem \textit{Tracer Study} Program Studi Ilmu Komputer FMIPA UNJ ke dalam program berbasis \textit{website} ?
\end{enumerate}

\section{Batasan Masalah}
Adapun batasan masalah pada penelitian ini ialah :
\begin{enumerate}
	\item Metode pengembangan sistem yang digunakan merupakan salah satu model pengembangan perangkat lunak dari metode \textit{System Development Life Cycle}, yaitu \textit{spiral model}.
	
	\item Data hasil \textit{tracer study} pada penelitian ini dideskripsikan dalam bentuk grafik dan tabel.
\end{enumerate}

\section{Tujuan Penelitian}
Penyusunan tugas akhir ini bertujuan untuk mengembangkan sistem informasi \textit{tracer study} pada prodi Ilmu Komputer FMIPA UNJ berdasarkan analisis dan evaluasi dari perbandingan sistem yang telah tersedia, yaitu Sistem Informasi \textit{Tracer Study} UNJ dan Sistem \textit{Tracer Study} Dikti.

\section{Manfaat Penelitian}
Adapun manfaat yang diharapkan dari penelitian ini adalah sebagai berikut :
	\begin{enumerate}
		\item Bagi Peneliti
		\begin{itemize}
			\item Memberikan gambaran mengenai pekerjaan yang akan digeluti setelah lulus dari program studi Ilmu Komputer
			\item Menambah pengetahuan dan keterampilan dalam mengembangkan suatu \textit{website}.
		\end{itemize}	
		\item Bagi Alumni, memberikan referensi terkait jenjang pekerjaan yang dapat digeluti oleh lulusan Ilmu Komputer. 
		\item Bagi Program Studi
		\begin{itemize}
			\item Mengetahui tingkat relevansi antara sistem pendidikan pada program studi Ilmu Komputer dengan kebutuhan di dunia kerja. 
			\item Mengetahui apakah kompetensi yang dimiliki lulusan ilmu komputer sesuai dengan yang diharapkan program studi. 
			\item Memberikan informasi terkait data akreditasi yang dibutuhkan oleh progam studi Ilmu Komputer. 
			\item Memberikan masukan kepada program studi terkait kualitas alumni dan kurikulum yang ada berdasarkan penilaian dan umpan balik dari pengguna alumni.
			\item Mengetahui sejauh mana daya serap program studi Ilmu Komputer pada lapangan pekerjaan. 
		\end{itemize}  	
		\item Bagi Masyarakat
		\begin{itemize}
			\item Memfasilitasi masyarakat untuk lebih mengenal prospek kerja dari prodi Ilmu Komputer UNJ dan kualitas lulusannya.
			\item Meningkatkan kepercayaan dan pengakuan kiprah lulusan prodi Ilmu Komputer UNJ di masyarakat
			\item Memberikan referensi mahasiswa dalam mencari tempat Praktik Kerja Lapangan 
			\item Memberikan bekal bagi mahasiswa agar setelah lulus dapat lebih siap untuk memasuki dunia kerja dan dapat beradaptasi dengan baik.
		\end{itemize} 
	\end{enumerate}
		
% Baris ini digunakan untuk membantu dalam melakukan sitasi
% Karena diapit dengan comment, maka baris ini akan diabaikan
% oleh compiler LaTeX.
\begin{comment}
\bibliography{daftar-pustaka}
\end{comment}
