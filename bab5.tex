%!TEX root = ./template-skripsi.tex
%-------------------------------------------------------------------------------
%                            	BAB V
%               		KESIMPULAN DAN SARAN
%-------------------------------------------------------------------------------

\chapter{KESIMPULAN DAN SARAN}

\section{Kesimpulan}
Berdasarkan hasil implementasi dan pengujian program ini, maka didapat kesimpulan sebagai berikut:

\begin{enumerate}
	\item Pengembangan Sistem Informasi \textit{Tracer Study} Program Studi Ilmu Komputer berdasarkan dari analisis dan evaluasi perbandingan fitur sistem \textit{tracer study} yang telah tersedia, yaitu Sistem Informasi \textit{Tracer Study} Universitas Negeri Jakarta dan Sistem \textit{Tracer Study} Dikti. Selain itu, melalui pengumpulan data yang diperoleh dari analisis kebutuhan prodi Ilmu Komputer. 
	
	\item Proses pengembangan Sistem Informasi \textit{Tracer Study} Program Studi Ilmu Komputer menggunakan metode pengembangan perangkat lunak \textit{System Development Life Cycle} dengan \textit{Spiral Model} yang memiliki beberapa tahapan, yaitu pengumpulan dan analisis kebutuhan, perancangan atau desain, konstruksi atau pengkodean, dan pengujian atau evaluasi.
	
	\item Sistem Informasi \textit{Tracer Study} Program Studi Ilmu Komputer dikembangkan dengan bahasa pemrograman PHP dengan bantuan \textit{framework Codeigniter}. %Berdasarkan hasi user acceptence test dengan 	menggunakan metode black box, didapatkan rata-rata nilai 4,29 untuk responden alumni dan 4,70 untuk responden admin program studi dalam skala 5. Maka dapat disimpulkan bahwa sistem informasi yang dikembangkan sudah sesuai dan dapat berjalan dengan baik.
	
	\item Berdasarkan hasil \textit{User Acceptance Test} pada pengujian fungsional didapatkan bahwa fitur-fitur yang terdapat pada Sistem Informasi \textit{Tracer Study} Program Studi Ilmu Komputer dapat berjalan dengan baik dan sesuai kebutuhan.
	
	\item Berdasarkan hasil \textit{User Acceptance Test} pada pengujian kebergunaan (\textit{usability}), didapatkan total persentase kelayakan dari keseluruhan sistem adalah 85.66\%. Nilai tersebut terdapat pada skor skala \textit{likert} 81\%-100\%, maka nilai kebergunaan Sistem Informasi \textit{Tracer Study} Program Studi Ilmu Komputer mendapat predikat sangat sesuai.
	
	\item  Sistem Informasi \textit{Tracer Study} Program Studi Ilmu Komputer dikembangkan untuk dapat dimanfaatkan pihak prodi terkait pengelolaan data alumni, menghasilkan informasi hasil \textit{tracer study} yang dibutuhkan prodi untuk mengingkatkan kualitas pendidikannya dan memenuhi kebutuhan akreditasi. Selain itu, ditujukan untuk masyarakat agar dapat lebih mengenal lulusan prodi Ilmu Komputer

\end{enumerate}


\section{Saran}
Adapun beberapa saran untuk penelitian selanjutnya adalah:
\begin{enumerate}
	
	\item Ruang lingkup penelitian selanjutnya diharapkan dapat ditingkatkan menjadi rumpun maupun fakultas. 
	
	\item Desain dan fungsionalitas sistem diharapkan lebih memperhatikan pada peramban ponsel (\textit{mobile friendly})
	
	\item Pada sistem diharapkan untuk menambahkan halaman yang menampilkan status pengisian kuesioner apakah pernah mengisi atau belum
	
	
\end{enumerate}


% Baris ini digunakan untuk membantu dalam melakukan sitasi
% Karena diapit dengan comment, maka baris ini akan diabaikan
% oleh compiler LaTeX.
\begin{comment}
\bibliography{daftar-pustaka}
\end{comment}